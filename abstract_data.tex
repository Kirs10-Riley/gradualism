\documentclass[12pt]{article}

\addtolength{\textwidth}{.6in}
\addtolength{\oddsidemargin}{-.3in}
\setlength{\textheight}{8.4in}
\setlength{\topmargin}{-.25in}
\setlength{\headheight}{0.0in}
\renewcommand{\baselinestretch}{1.0}
\linespread{1.15}

\usepackage{cite}
\usepackage{times, verbatim,xcolor,bm}
\usepackage{amsbsy,amssymb, amsmath, amsthm}
\usepackage{hyperref}

\newtheorem{definition}{Definition}
\newtheorem{theorem}{Theorem}
\newtheorem{lemma}{Lemma}
\newtheorem{corollary}{Corollary}
\newtheorem{assumption}{Assumption}
\newtheorem{fact}{Fact}

\newcommand{\ve}{\varepsilon}
\newcommand{\ov}{\overline}
\newcommand{\un}{\underline}
\newcommand{\ta}{\theta}
\newcommand{\Ta}{\Theta}
\newcommand{\expect}{\mathbb{E}}

\begin{document}
\title{\vskip-0.6in Quantifying GATT Trade Liberalization}
\author{Kristy Buzard and Zeyuan Xiong}
\date{\today}
\maketitle

Large-scale tariff reductions within the framework of the GATT/WTO have come as a result of a series of nine rounds of negotiations that began in 1947. 

The literature that attempts to explain this gradual reduction in trade barriers over time is almost completely theoretical. 

In Devereaux (1997), increasing benefits of integration to consumers gradually increase the costs of trade wars and lead to free trade over time. Similarly, Chisik's (2003) assumption that capacity accumulation in the export sector is partially irreversible leads to a gradually increasing dependence of export producers on trade and therefore increases countries' incentives to lower tariffs in successive negotiating rounds. On the import-competing side, Staiger's (1995) model focuses on reductions in the size of the import-competing sector, with gradualism arising from the presence of workers with specialized skills that allow them to earn rents in the import-competing industry. In the context of unilateral trade opening, Mehlum (1998) demonstrates that gradual tariff reductions can improve welfare in the presence of a minimum wage, whereas Mussa (1986) shows similar results by assuming the presence of adjustment costs that are convex in the number of workers leaving the import-competing sector. Furusawa and Lai (1999) show a similar result in the context of an infinitely repeated tariff setting game between governments, and Zissimos (2007) demonstrates that the GATT requirement that punishments be limited to the `withdrawal of equivalent concessions' generates gradualism.

----

Data availability has been a major barrier to research on this massive dismantling of trade barriers. Bown and Irwin (2017) have documented the available data, finding it to be frustratingly sparse. They find data on applied tariffs, i.e. the tariff levels that are charged as opposed to the upper bound on tariffs that is negotiated, and never disaggregated below the level of ten broad sectors. 

As little as we know about applied tariffs in this context, we know even less about negotiated tariffs. Somewhat surprisingly, data on the tariff commitments made by the GATT signees has not been widely available, preventing detailed studies on the workings of the GATT and the dynamics of this important episode of trade liberalization.

Both having more industrial detail about tariffs and having this detail on the negotiated tariffs is important. First, the negotiated tariff commitments allow us to study the political constraints that led to the GATT agreements. The negotiated tariffs are the cleanest data that exist for all negotiating parties since they are not complicated by the various processes that determine applied tariffs. Second, often the most important object in terms of policymaking is the gap between the negotiated tariff and the applied tariff, the so-called “water” in the bindings. We can’t know how much water is in the bindings if we don’t know what the negotiated tariff caps were.

---

The original intention in the ‘Gradualism’ project was to provide supportive evidence for the theory using a sample of historical data from the early rounds of the GATT negotiations. However, I quickly discovered—as mentioned above—that this data is not available. Talking to the leading experts in the field confirmed that there is an egregious gap in knowledge simply about what actually happened during the early GATT. 

I’ve been able to find original documents that contain the GATT tariff schedules at the end of each of the nine round of negotiations. Some of these were digital copies, and some we could find only in hard copy. We have begun a process of digitalizing and standardizing the data for the first few rounds, and we have a tentative data structure in place that we can adapt as we bring in data from the newer rounds.

It is therefore time to forge ahead with completing the digitalization and standardization of the data.

\vskip.2in
Questions for analysis stage
\begin{enumerate}
	\item Which products/industries suffer from large tariff cuts and which industries/products continue to receive protection?
	\item What was the total reduction in negotiated tariffs under the GATT (relative to the 1930 Smoot-Hawley tariffs for the U.S.) in each round? That is, just what did the GATT accomplish?
	\item Were tariffs cut gradually across the board, or was there variation in the speed with which tariffs were cut across products? If there is variation, can we find an explanation for that variation?
	\item Which types of products had ad-valorem versus specific tariffs, and how did this change over time? What role did the presence of specific tariffs, combined with inflation, have in reducing the total level of tariff protection?
\end{enumerate}

\vskip.2in
Questions for after more data work is completed
\begin{enumerate}
	\item How does the number of countries who are negotiating relate to the number of lines that are included?
	\item Can we just look at the number of pages for the schedule for each country?
\end{enumerate}
				

\newpage
\noindent\large\textbf{References}\\

\normalsize \noindent Bown, C. P. and D. A. Irwin, ``The GATT's Starting Point: Tariff Levels circa 1947,'' in Manfred Elsig, Bernard Hoekman, and Joost Pauwelyn (eds.), Assessing the World Trade Organization: Fit for Purpose? Cambridge, UK: Cambridge University Press, 2017, 45-74 (Chapter 3).\\

\noindent Chisik, R., 2003. ``Gradualism in free trade agreements: a theoretical justification.'' Journal of International Economics, 59, 367-397. \\

\noindent Devereux, M., 1997. ``Growth, specialization, and trade liberalization.'' International Economic Review
38, 565-585. \\

\noindent Furusawa, T., Lai, E., 1999. ``Adjustment costs and gradual trade liberalization.'' Journal of International
Economics 49, 333-361. \\

\noindent Mehlum, H., 1998. ``Why gradualism?'' The Journal of International Trade and Economic Development 7, 279-297. \\

\noindent Mussa, M., 1986. ``The adjustment process and the timing of trade liberalization.'' In: Choksi, A., Papageorgiou, D. (Eds.), Economic Liberalization in Developing Countries. Basil Blackwell, Oxford. \\

\noindent Staiger, R., 1995. ``A theory of gradual trade liberalization.'' In: Levinsohn, J., Deardorff, A., Stern, R.
(Eds.), New Directions in Trade Theory. University of Michigan Press, Ann Arbor, MI, pp. 249-284. \\

\noindent Zissimos, B, 1997. ``The GATT and gradualism.'' The Journal of International Economics, 71, 410-433.\\

\end{document}