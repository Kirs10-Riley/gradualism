\documentclass[12pt]{article}

\addtolength{\textwidth}{.6in}
\addtolength{\oddsidemargin}{-.3in}
\setlength{\textheight}{8.4in}
\setlength{\topmargin}{-.25in}
\setlength{\headheight}{0.0in}
\renewcommand{\baselinestretch}{1.0}
\linespread{1.15}

\usepackage{cite}
\usepackage{times, verbatim,xcolor,bm}
\usepackage{amsbsy,amssymb, amsmath, amsthm}
\usepackage{hyperref}

\newtheorem{definition}{Definition}
\newtheorem{theorem}{Theorem}
\newtheorem{lemma}{Lemma}
\newtheorem{corollary}{Corollary}
\newtheorem{assumption}{Assumption}
\newtheorem{fact}{Fact}

\newcommand{\ve}{\varepsilon}
\newcommand{\ov}{\overline}
\newcommand{\un}{\underline}
\newcommand{\ta}{\theta}
\newcommand{\Ta}{\Theta}
\newcommand{\expect}{\mathbb{E}}

\begin{document}
\title{\vskip-0.6in Quantifying GATT Trade Liberalization}
\author{Kristy Buzard and Zeyuan Xiong}
\date{\today}
\maketitle

\section{Motivation}
Large-scale tariff reductions within the framework of the General Agreement on Tariffs and Trade (GATT) have come as a result of a series of eight rounds of negotiations that began in 1947. Most of the literature that attempts to understand this phenomenon is theoretical,\footnote{See, for instance, the literature that seeks to explain the gradual nature of this liberalization, e.g. Devereaux (1997) and  Chisik (2003) who focus on exporters; Mussa (1986), Staiger (1995) or Mehlum (1998) on the import-competing story; or Zissimos (2007) on the role of the `withdrawal of equivalent concessions' rule.} in large part because data availability has been a major barrier to empirical research on this massive dismantling of trade barriers. Bown and Irwin (2017) have documented the available data, finding it to be frustratingly sparse. They find data on the tariffs that are actually charged at the border---i.e. applied tariffs. These are at most disaggregated to the level of ten broad sectors. 

As little as we know about applied tariffs in this context, we know even less about the negotiated tariff bindings. Somewhat surprisingly, data on the tariff commitments made by the GATT signees has not been widely available, preventing empirical studies on the workings of the GATT and the dynamics of this important episode of trade liberalization. One notable exception {\color{red}[add here on Bagwell, Staiger and Yurukoglu]}.

Both having more industrial detail about tariffs and having this detail on the negotiated tariffs is important. First, the negotiated tariff commitments allow us to study the political constraints that led to the GATT agreements. The negotiated tariffs are the cleanest data that exist for all negotiating parties since they are not complicated by the various processes that determine applied tariffs. Second, often the most important object in terms of policymaking is the gap between the negotiated tariff and the applied tariff, the so-called “water” in the bindings {\color{red} [Add citations here--Kuenzel, Pelc, Busch}. Even if we know what the applied tariffs were, we can’t know how much water is in the bindings if we don’t also know the negotiated tariff caps.

\section{What we do}
We have been able to locate the original documents that contain the consolidated GATT tariff schedules at the end of each of the nine round of negotiations.\footnote{\color{red} Some of these were digital copies, and some we could find only in hard copy.} To date, we have digitalized and standardized the tariff data for the United States for the first five GATT rounds for the United States. We have done the same for the tariff schedule that was in effect before the start of the GATT---the so-called Smoot Hawley tariffs.

Work is underway to add the Kennedy, Tokyo and Uruguay rounds for the U.S. and all rounds for several other countries. Our preliminary findings reported herein are for the U.S. data on Smoot Hawley tariffs through the Dillon round.

\vskip.4in
\begin{enumerate}
	\item What was the total reduction in negotiated tariffs under the GATT (relative to the 1930 Smoot-Hawley tariffs for the U.S.) in each round? That is, just what did the GATT accomplish?
	\item Which products/industries suffer from large tariff cuts and which industries/products continue to receive protection?
	\item Were tariffs cut gradually across the board, or was there variation in the speed with which tariffs were cut across products? If there is variation, can we find an explanation for that variation?
	\item Which types of products had ad-valorem versus specific tariffs, and how did this change over time? What role did the presence of specific tariffs, combined with inflation, have in reducing the total level of tariff protection?
\end{enumerate}


%\vskip.2in
%Questions for after more data work is completed
%\begin{enumerate}
%	\item How does the number of countries who are negotiating relate to the number of lines that are included?
%	\item What oculd we infer by looking at the number of pages for the schedule for each country?
%\end{enumerate}
				

\newpage
\noindent\large\textbf{References}\\

\normalsize \noindent Bown, C. P. and D. A. Irwin, ``The GATT's Starting Point: Tariff Levels circa 1947,'' in Manfred Elsig, Bernard Hoekman, and Joost Pauwelyn (eds.), Assessing the World Trade Organization: Fit for Purpose? Cambridge, UK: Cambridge University Press, 2017, 45-74 (Chapter 3).\\

\noindent Chisik, R., 2003. ``Gradualism in free trade agreements: a theoretical justification.'' Journal of International Economics, 59, 367-397. \\

\noindent Devereux, M., 1997. ``Growth, specialization, and trade liberalization.'' International Economic Review
38, 565-585. \\

\noindent Furusawa, T., Lai, E., 1999. ``Adjustment costs and gradual trade liberalization.'' Journal of International
Economics 49, 333-361. \\

\noindent Mehlum, H., 1998. ``Why gradualism?'' The Journal of International Trade and Economic Development 7, 279-297. \\

\noindent Mussa, M., 1986. ``The adjustment process and the timing of trade liberalization.'' In: Choksi, A., Papageorgiou, D. (Eds.), Economic Liberalization in Developing Countries. Basil Blackwell, Oxford. \\

\noindent Staiger, R., 1995. ``A theory of gradual trade liberalization.'' In: Levinsohn, J., Deardorff, A., Stern, R.
(Eds.), New Directions in Trade Theory. University of Michigan Press, Ann Arbor, MI, pp. 249-284. \\

\noindent Zissimos, B, 1997. ``The GATT and gradualism.'' The Journal of International Economics, 71, 410-433.\\

\end{document}